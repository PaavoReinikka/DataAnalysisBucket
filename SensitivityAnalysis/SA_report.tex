\documentclass[]{article}
\usepackage{lmodern}
\usepackage{amssymb,amsmath}
\usepackage{ifxetex,ifluatex}
\usepackage{fixltx2e} % provides \textsubscript
\ifnum 0\ifxetex 1\fi\ifluatex 1\fi=0 % if pdftex
  \usepackage[T1]{fontenc}
  \usepackage[utf8]{inputenc}
\else % if luatex or xelatex
  \ifxetex
    \usepackage{mathspec}
  \else
    \usepackage{fontspec}
  \fi
  \defaultfontfeatures{Ligatures=TeX,Scale=MatchLowercase}
\fi
% use upquote if available, for straight quotes in verbatim environments
\IfFileExists{upquote.sty}{\usepackage{upquote}}{}
% use microtype if available
\IfFileExists{microtype.sty}{%
\usepackage{microtype}
\UseMicrotypeSet[protrusion]{basicmath} % disable protrusion for tt fonts
}{}
\usepackage[margin=1in]{geometry}
\usepackage{hyperref}
\hypersetup{unicode=true,
            pdftitle={Sensitivity Analysis on oat data (ELY keskus)},
            pdfauthor={Paavo Reinikka},
            pdfborder={0 0 0},
            breaklinks=true}
\urlstyle{same}  % don't use monospace font for urls
\usepackage{graphicx,grffile}
\makeatletter
\def\maxwidth{\ifdim\Gin@nat@width>\linewidth\linewidth\else\Gin@nat@width\fi}
\def\maxheight{\ifdim\Gin@nat@height>\textheight\textheight\else\Gin@nat@height\fi}
\makeatother
% Scale images if necessary, so that they will not overflow the page
% margins by default, and it is still possible to overwrite the defaults
% using explicit options in \includegraphics[width, height, ...]{}
\setkeys{Gin}{width=\maxwidth,height=\maxheight,keepaspectratio}
\IfFileExists{parskip.sty}{%
\usepackage{parskip}
}{% else
\setlength{\parindent}{0pt}
\setlength{\parskip}{6pt plus 2pt minus 1pt}
}
\setlength{\emergencystretch}{3em}  % prevent overfull lines
\providecommand{\tightlist}{%
  \setlength{\itemsep}{0pt}\setlength{\parskip}{0pt}}
\setcounter{secnumdepth}{0}
% Redefines (sub)paragraphs to behave more like sections
\ifx\paragraph\undefined\else
\let\oldparagraph\paragraph
\renewcommand{\paragraph}[1]{\oldparagraph{#1}\mbox{}}
\fi
\ifx\subparagraph\undefined\else
\let\oldsubparagraph\subparagraph
\renewcommand{\subparagraph}[1]{\oldsubparagraph{#1}\mbox{}}
\fi

%%% Use protect on footnotes to avoid problems with footnotes in titles
\let\rmarkdownfootnote\footnote%
\def\footnote{\protect\rmarkdownfootnote}

%%% Change title format to be more compact
\usepackage{titling}

% Create subtitle command for use in maketitle
\providecommand{\subtitle}[1]{
  \posttitle{
    \begin{center}\large#1\end{center}
    }
}

\setlength{\droptitle}{-2em}

  \title{Sensitivity Analysis on oat data (ELY keskus)}
    \pretitle{\vspace{\droptitle}\centering\huge}
  \posttitle{\par}
    \author{Paavo Reinikka}
    \preauthor{\centering\large\emph}
  \postauthor{\par}
      \predate{\centering\large\emph}
  \postdate{\par}
    \date{27 August 2019}


\begin{document}
\maketitle

\subsection{Setup for SA}\label{setup-for-sa}

Sensitivity analysis was performed on full oat data, and the focus of
the analysis was on the input-output sensitivity of various elnet models
(as defined and implemented in the r package \textbf{glmnet}). The
models were differentiated on two axis: \emph{alpha} being the tuning
parameter for the continuum Ridge - Lasso, and \emph{lambda} being the
parameter deciding the amount of penalization (regularization). The
parameters were defined on a discrete grid: \(\lambda \cup \alpha\) st.
\(\lambda \in [min, 1se]\) and \(\alpha \in [0,0.25,0.5,0.75,1]\), where
\emph{lambda} value min corresponds to model with minimum cv error (per
\emph{alpha}) and value 1se corresponds to maximum cv error
\emph{within} 1 se from that minimum. Different model runs are therefore
consistent not by the exact values of lambda, but rather the heuristic
for choosing the lambda. This affords us some added aggregation over the
data.

The SA performed on the oat data varied only predictors (input-output
SA), and only after model had been built; i.e., effects of
variations/perturbations in the training data on the modelling (building
of the model) were not observed. Also, modelling assumptions were not
varied directly, as a source of output variation. However, some insights
to the effects of the main elnet parameters were acquired indirectly by
performing input-output SA along different values of alpha, lambda.

\subsection{Used measures and their
limitations}\label{used-measures-and-their-limitations}

In the SA literature there are many different sensitivity measures, both
local and global, with varying degrees of constraints. Use of
\emph{sigma-normalized coefficients} (Saltelli et.al.{[}1{]}) are
considered a sufficient measure when model is known to be linear and
predictors un-correlated. In a more general setting however, when a
certain level of agnostisism is preferred with regards to the model and
the input space (e.g., ML setting), MC sampling coupled with either
variance or enthropy based measures are preferred. SA here is performed
using r package \emph{sensitivity} and as sufficient measures Sobol's
1st and total order indices{[}2{]}.

For model y = f(\(\sf{x_{1}},..., x_{p}\)) where y is a scalar
(`production') and x's are the columns of the input matrix, the 1st
order sensitivity index, corresponding to \emph{main effects} of
\(\sf{x_{i}}\), is defined:
\[\sf{S_{i}}=\dfrac{V_{x_{i}}(E_{X_{-i}}(y|x_{i}))}{V(y)}\] where the
inner (\(\sf{x_{i}}\) conditional) expectation is taken over all other
predictors, and the outer variance over \(\sf{x_{i}}\). Similarly the
total order index, which corresponds to all other effects of
\(\sf{x_{i}}\) (through it's interactions with other variables) can be
defined:
\[\sf{ST_{i}}=\dfrac{E_{X_{-i}}(V_{x_{i}}(y|X_{-i}))}{V(y)}=1-\dfrac{V_{X_{-i}}(E_{x_{i}}(y|X_{-i}))}{V(y)}\]

These indices are often presented as for independent input
distributions, when the property \(0\leq\sf{S_{i}}\leq\sf{ST_{i}}\leq1\)
holds.

\subsection{Relevant literature}\label{relevant-literature}

{[}1{]}Main source by Saltelli et. al.
\url{http://www.andreasaltelli.eu/file/repository/A_Saltelli_Marco_Ratto_Terry_Andres_Francesca_Campolongo_Jessica_Cariboni_Debora_Gatelli_Michaela_Saisana_Stefano_Tarantola_Global_Sensitivity_Analysis_The_Primer_Wiley_Interscience_2008_.pdf}

{[}2{]}I M Sobol'. Sensitivity estimates for nonlinear mathematical
models.Mathematical Modelling andComputational Experiments, 1:407--414,
1993.


\end{document}
